\documentclass[12pt]{article}
\usepackage[spanish]{babel}

\title{Tarea 2}
\author{Emilio Junoy de Juambelz}
\date{\today}

\begin{document}
  \maketitle
  \begin{enumerate}
    \item[1.] Para un alfabeto de 26 letras, ¿Cuántas matrices de 2x2 hay que nos permitan cifrar mediante Hill?
      Justifica tu repuesta.\\
      Para que una matriz de 2x2 pueda ser utilizada para cifrar con el método de Hill, se necesita que 
      ésta sea invertible módulo el tamaño del alfabeto, es decir 26. Para que una matriz sea invertible 
      módulo 26 necesitamos que su determinante sea un primo relativo a 26. Podemos hacer un pequeño programa de Python
      que implemente el algoritmo de Euclides para calcular el máximo común divisor entre los posibles determinantes de 
      las matrices de 2x2 y el 26, y así determinar el número de matrices de 2x2 invertibles módulo 26.\\
      Realizando esto, vemos que hay un total de 163535 matrices que podemos usar para cifrar con el método de Hill,
      sin embargo, una de estas matrices es la identidad, asì que tenemos 163535 matrices para cifrar no trivialmente 
      con el método de Hill.
    
    \item[2.] Investiga un caso de la vida real donde se rompió la seguridad con máximas de Kerckoffs.\\
      Un caso muy importante en la historia es el de la máquina Enigma. Recordemos que una de las máximas de Kerckoffs 
      es que la seguridad no debe residir en la obsuridad del método, sino en la llave utilizada. Esta máxima fue rota
      por los Alemanes en la segunda guerra mundial. Las claves en la máquina Enigma eran las posiciones en las que se
      colocaban los rotores, pero los Alemanes frecuentaban usar la misma clave o usar claves con patrones predecibles, 
      lo que llevó a que los Aliados lograran descifrar el código Enigma.
      Así, las llaves que elegimos para cifrar 
      son fundamentales para la seguridad del método, pues quebrantar esta máxima de Kerckoffs puede tener 
      consecuencias irremediables.

    \item[3.] Usando el polinomio $x^4+x+1$, da la lista de bits de salida de un LFSR asociado usando como semilla 
      una secuencia de ceros, y cualquier otra que escojas.

    \item[4.] Supón que se manda un criptotexto $c=c_1,c_2, \ldots$, pero se pierde el bloque $c_2$, así que 
      se recibe $c=c_1, c_3, \ldots$. Al descifrar el mensaje, ¿Cuál es el efecto de un bloque no recibido al usar los modos 
      de operación CBC, OFB y CTR?\\
      Si usamos el modo CBC, el mensaje claro $m_i$ depende de los textos cifrados $c_i$ y $c_{i+1}$ de la siguiente manera 
      $m_i = D_K(c_{i})\oplus c_{i-1}$. Si se pierde el bloque $c_2$ esto afectará a los mensajes claros $m_2$ y $m_3$, el mensaje $m_2$ 
      lo descifraríamos como $m_2 = D_K(c_{3})\oplus c_{1}$, lo que no tiene sentido. El mensaje $m_3$ lo descifraríamos como $m_3 = D_K(c_{4})\oplus c_3 = m_4$, 
      de esta manera, lo que pasaría es que sólo el mensaje $m_2$ dejaría de tener sentido y el mensaje $m_3$ no se recibe, los restantes mensajes se descifrarían de manera correcta.\\
      Si usamos el modo OFB, el mensaje claro $m_i$ depnde de los anteriores mensajes de la siguiente manera 
      $m_i = c_i \oplus E_K(c_{i-1}\oplus E_K(c_{i-2}\oplus \cdots))$. Asì, si se pierde el bloque $c_2$ el único bloque de texto que podríamos descifrar correctamente es el primero.\\
      Si usamos el modo CTR, también el único mensaje que podemos descifrar correctamente es el primero, pues en los restantes, el contador que usamos para descifrar el 
      bloque $c_i$ es en realidad el correspondiente al bloque $c_{i+1}$.
    \item[5.] Investiga qué es el cifrado RC4. Da la descripción de su funcionamiento y sus debilidades.

    \item[6.] Determina si $a$ es residuo cuadrático módulo $n$. Muestra tu procedimiento
  \end{enumerate}
\end{document}

